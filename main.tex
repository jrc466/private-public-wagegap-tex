\documentclass{article}
\usepackage[utf8]{inputenc}
\usepackage{natbib}

\title{Public-private wage gap in Brazil: a counterfactual analysis using longitudinal data}
\author{Vítor Costa}
\date

\begin{document}

\maketitle
\tableofcontents

\section{Introduction and motivation}
Anecdotal evidence in Brazil suggests public sector workers are overcompensated compared to their private sector counterparts. In fact, (brief list of references for the Brazilian economy and their estimates). (Highlight the discrepancy of these estimates). (highlight their lack of treatment for endogenous selection).

Add
\begin{enumerate}
    \item Segmentation (the two different legal frameworks, CLT vs Statutory)
    \item Endogenous selection (selection via highly competitive civil exams)
    \item Increasing share of personnel expense in the national budget 
    \item Results for other countries
\end{enumerate}


This paper's contribution to this literature is twofold: I apply a quantile Oaxaca-Blinder decomposition method and use the panel structure of the data to estimate worker fixed effects. The method allows me to estimate composition and structural effects of the private-public gap along the distribution of wages. Endogeneity procedure relies on \cite{canay_simple_2011}.  

Add 
\begin{enumerate}
    \item conclusions
    \item sections layout 
\end{enumerate}

\section{Previous Evidence}

\section{Brazilian Civil Service Panorama}

\section{Data}
\subsection{RAIS}
\begin{enumerate}
    \item Filter RAIS for: prime age workers (25-54), keep only CNPJs - drop CEIs, drop blank PIS, select only federal administration employees, keep only clt and statutory.
    \item sample using 1\% or 10\% of whole data set.  
    \item validate CNPJ and PIS
    \item compute log hourly wages (hired wages over hired hours)
    \item select contract types (permanent vs temporary)
    \item create variable for low, medium and high skills
    \item create different subsets from sample, skills $\times$ sex $\times$ race.
    \item show descriptive statistics.
\end{enumerate}

\subsection{Comparison with data in previous papers}
\begin{enumerate}
    \item describe advantages over PNAD
    \item describe advantages over PME
\end{enumerate}

\section{Equations for estimations}
\subsection{Naïve equations}
\begin{enumerate}
    \item Describe the Oaxaca-Blinder decomposition and outline the equation
    \item Describe the Quantile OB, methodology \cite{chernozhukov_inference_2013} 
\end{enumerate}
\subsection{Accounting for endogenous selection}
\begin{enumerate}
    \item Discuss \cite{canay_simple_2011} and outline equation for worker efffects estimation
    \item Outline net of FEs equation 
\end{enumerate}

\section{Estimation Results}
\subsection{Oaxaca-Blinder (OB)}
\begin{enumerate}
    \item Report overall OB 
    \item Plot yearly OB in both cases
\end{enumerate}
\subsection{Distribution of the Premium}
\begin{enumerate}
    \item Plot overall distribution in two cases
    \item Show distribution estimates broken by demographic groups
\end{enumerate}

\section{Conclusion}
 \begin{enumerate}
     \item what do we learn from the quantile decomposition? where does the premium phase out?
     \item is self-selection important? by how much?
     \item what is the size of premium once we account for selection?
     \item show expenses with inactive civil employees and its evolution in time.
     \item state the discrepancy of retirement entitlements for both types of workers and problems for future research. 
\end{enumerate}  

\newpage
\scriptsize{\bibliographystyle{chicago}}
\bibliography{library} %bibtex file name without .bib

\end{document}
